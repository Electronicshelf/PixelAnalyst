\chapter{Local Dimming and Blooming Control}
\section{FALD vs Edge-Lit}
\begin{table}[H]
\centering
\caption{Local Dimming Comparison}
\begin{tabular}{@{}lll@{}}
\toprule
\textbf{Feature} & \textbf{FALD} & \textbf{Edge-Lit} \\
\midrule
Zone count & 100--1000+ & 8--32 \\
Contrast & 10,000--50,000:1 & 1,000--5,000:1 \\
Blooming & Low & High \\
Cost & High & Low \\
Thickness & Thicker & Thinner \\
\bottomrule
\end{tabular}
\end{table}

\section{Zone-Based Luminance Model}
\begin{formulabox}{Backlight Convolution Model}
For local dimming LCDs, output luminance is approximately:
\begin{equation}
L(x,y,t) \approx \sum_{z=1}^{N_z} h_z(x,y)\,B_z(t)\,T(x,y,t) + L_{\text{leak}}
\end{equation}
where:
\begin{itemize}
  \item $h_z(x,y)$ = optical spread function of zone $z$
  \item $B_z(t)$ = zone backlight command
  \item $T(x,y,t)$ = LCD transmittance for pixel $(x,y)$
  \item $L_{\text{leak}}$ = panel leakage floor
\end{itemize}
\end{formulabox}

\begin{importantbox}{Blooming Trade-off}
Aggressive zone boosting increases highlight detail but also raises halo artifacts because each zone PSF spreads into neighboring dark pixels. Practical algorithms therefore include temporal and spatial regularization terms.
\end{importantbox}

