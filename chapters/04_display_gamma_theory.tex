\part{Gamma and Transfer Functions}

\chapter{Display Gamma Theory}

\section{The Pure Power Law}

\begin{formulabox}{Fundamental Gamma Relationship}
\begin{equation}
  L = L_{\max} V^{\gamma}
\end{equation}
where:
\begin{itemize}
  \item $L$ = output luminance [cd/m²]
  \item $L_{\max}$ = maximum display luminance
  \item $V$ = normalized input signal ($0 \le V \le 1$)
  \item $\gamma$ = gamma exponent (typically 2.2 to 2.5)
\end{itemize}
\end{formulabox}

\subsection{Historical Origin: CRT Physics}

The gamma function originates from cathode ray tube (CRT) physics:

\begin{equation}
  L \propto I \propto V_g^{3/2} \times \text{(additional factors)} \approx V_g^{2.5}
\end{equation}

where $I$ is electron beam current and $V_g$ is grid voltage.

\begin{figure}[H]
\centering
\begin{tikzpicture}
  \begin{axis}[
    width=12cm,
    height=8cm,
    xlabel={Input Signal $V$},
    ylabel={Output Luminance (normalized)},
    xmin=0, xmax=1,
    ymin=0, ymax=1,
    grid=major,
    legend pos=north west,
    legend style={fill=white, fill opacity=0.8}
  ]
  
  \addplot[color=primarycolor, very thick, domain=0:1, samples=100] {x};
  \addlegendentry{$\gamma = 1.0$ (linear)}
  
  \addplot[color=successcolor, very thick, domain=0:1, samples=100] {x^1.8};
  \addlegendentry{$\gamma = 1.8$}
  
  \addplot[color=accentcolor, very thick, domain=0:1, samples=100] {x^2.2};
  \addlegendentry{$\gamma = 2.2$ (sRGB)}
  
  \addplot[color=warningcolor, very thick, domain=0:1, samples=100] {x^2.4};
  \addlegendentry{$\gamma = 2.4$ (BT.1886)}
  
  \addplot[color=codecolor, very thick, domain=0:1, samples=100] {x^2.6};
  \addlegendentry{$\gamma = 2.6$}
  
  \end{axis}
\end{tikzpicture}
\caption{Gamma curves for different exponent values. Higher gamma produces darker midtones.}
\label{fig:gamma_curves}
\end{figure}

\section{Bit Depth and Quantization}

\begin{formulabox}{Perceptual Bit Depth Requirement}
Number of JNDs (Just Noticeable Differences) across range $[L_{\min}, L_{\max}]$:
\begin{equation}
  N_{\text{JND}} = \frac{\ln(L_{\max}/L_{\min})}{\ln(1 + \Delta L/L)}
\end{equation}

For SDR (0.01 to 100 cd/m²) with $\Delta L/L = 0.01$:
\begin{equation}
  N_{\text{JND}} = \frac{\ln(10000)}{\ln(1.01)} \approx 924
\end{equation}

Required code values (with safety margin):
\begin{equation}
  N_{\text{codes}} = 1.5 \times N_{\text{JND}} \approx 1386
\end{equation}

Minimum bit depth:
\begin{equation}
  b = \log_2(1386) \approx 10.4 \text{ bits}
\end{equation}

\textbf{Therefore: 10-bit minimum for SDR without visible banding.}
\end{formulabox}

\begin{importantbox}{Gamma Encoding Benefit}
Gamma encoding increases effective bit depth:
\begin{equation}
  b_{\text{effective}} \approx b \times \gamma
\end{equation}

Thus 8-bit gamma-encoded ($\gamma = 2.2$):
\begin{equation}
  b_{\text{effective}} \approx 8 \times 2.2 = 17.6 \text{ effective linear bits}
\end{equation}

This explains why 8-bit gamma-encoded images appear smooth while 8-bit linear shows severe banding. Gamma encoding is essentially a form of perceptually-motivated compression.
\end{importantbox}

\section{Standard Transfer Functions}

\subsection{sRGB Standard}

\begin{formulabox}{sRGB EOTF (Electro-Optical Transfer Function)}
\begin{equation}
  L = \begin{cases}
    V / 12.92 & \text{if } V \le 0.04045 \\
    \left[\frac{V + 0.055}{1.055}\right]^{2.4} & \text{otherwise}
  \end{cases}
\end{equation}

\textbf{sRGB OETF (Opto-Electronic Transfer Function):}
\begin{equation}
  V = \begin{cases}
    12.92 \times L & \text{if } L \le 0.0031308 \\
    1.055 \times L^{1/2.4} - 0.055 & \text{otherwise}
  \end{cases}
\end{equation}

\textbf{Properties:}
\begin{itemize}
  \item Linear segment near black avoids numerical instability
  \item Power segment uses $\gamma = 2.4$
  \item Effective average gamma $\approx 2.2$
  \item $C^1$ continuous (both value and derivative match at transition)
\end{itemize}
\end{formulabox}

\begin{figure}[H]
\centering
\begin{tikzpicture}
  \begin{axis}[
    width=12cm,
    height=8cm,
    xlabel={Input Signal $V$},
    ylabel={Output Luminance $L$},
    xmin=0, xmax=1,
    ymin=0, ymax=1,
    grid=major,
    legend pos=north west
  ]
  
  % sRGB EOTF
  \addplot[color=primarycolor, very thick, domain=0:0.04045, samples=50] {x/12.92};
  \addplot[color=primarycolor, very thick, domain=0.04045:1, samples=200] 
    {((x+0.055)/1.055)^2.4};
  \addlegendentry{sRGB EOTF}
  
  % Pure gamma 2.2 for comparison
  \addplot[color=accentcolor, dashed, thick, domain=0:1, samples=100] {x^2.2};
  \addlegendentry{Pure $\gamma = 2.2$}
  
  \end{axis}
\end{tikzpicture}
\caption{sRGB EOTF compared to pure power law. Note the linear segment near black.}
\label{fig:srgb_eotf}
\end{figure}

\subsection{BT.1886 Reference EOTF}

\begin{formulabox}{BT.1886 for Real Displays}
Real displays have non-zero black level. BT.1886 accounts for this:
\begin{equation}
  L = (aV + b)^\gamma
\end{equation}
where:
\begin{align}
  a &= L_{\max}^{1/\gamma} - L_{\min}^{1/\gamma} \\
  b &= L_{\min}^{1/\gamma}
\end{align}

\textbf{Verification of boundary conditions:}
\begin{align}
  L(0) &= b^\gamma = L_{\min} \quad \checkmark \\
  L(1) &= (a + b)^\gamma = L_{\max} \quad \checkmark
\end{align}

\textbf{Recommended gamma:} $\gamma = 2.4$ (home viewing, dim surround)
\end{formulabox}

\begin{examplebox}{Example: Laptop Display}
\textbf{Specifications:}
\begin{itemize}
  \item $L_{\max} = 500$ nits
  \item $L_{\min} = 0.5$ nits (LCD backlight leakage)
  \item $\gamma = 2.4$
\end{itemize}

\textbf{Calculate parameters:}
\begin{align}
  a &= 500^{1/2.4} - 0.5^{1/2.4} = 10.67 - 0.808 = 9.86 \\
  b &= 0.5^{1/2.4} = 0.808
\end{align}

\textbf{Full EOTF:}
\begin{equation}
  L = (9.86V + 0.808)^{2.4}
\end{equation}

\textbf{Verification:}
\begin{align}
  V = 0: \quad L &= (0.808)^{2.4} = 0.5 \text{ nits} \quad \checkmark \\
  V = 1: \quad L &= (10.67)^{2.4} = 500 \text{ nits} \quad \checkmark
\end{align}
\end{examplebox}


