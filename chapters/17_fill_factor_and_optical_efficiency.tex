\chapter{Fill Factor and Optical Efficiency}

\section{Definitions}

\begin{formulabox}{Key Metrics}
\textbf{Fill Factor:}
\begin{equation}
  \text{FF} = \frac{A_{\text{active}}}{A_{\text{total}}}
\end{equation}

\textbf{Aperture Ratio (LCD):}
\begin{equation}
  \text{AR} = \frac{A_{\text{transmitting}}}{A_{\text{total}}}
\end{equation}

\textbf{Typical Values:}
\begin{itemize}
  \item LCD: FF = 60--70\%, AR = 5--8\%
  \item OLED: FF = 70--85\%, AR = 100\% (emissive)
\end{itemize}
\end{formulabox}

\begin{figure}[H]
\centering
\begin{tikzpicture}[scale=1.2]
  \begin{axis}[
    width=12cm,
    height=7cm,
    xlabel={Pixel Pitch ($\mu$m)},
    ylabel={Fill Factor (\%)},
    xmin=0, xmax=100,
    ymin=50, ymax=90,
    grid=major,
    legend pos=south east
  ]
  
  % LCD fill factor decreases with smaller pixels
  \addplot[color=primarycolor, very thick, domain=20:100] 
    {85 - 500/x};
  \addlegendentry{OLED}
  
  \addplot[color=accentcolor, very thick, domain=20:100] 
    {75 - 400/x};
  \addlegendentry{LCD}
  
  \end{axis}
\end{tikzpicture}
\caption{Fill factor versus pixel pitch. Smaller pixels have lower fill factors due to fixed-size gaps and structures.}
\end{figure}

% Continue with many more chapters...

