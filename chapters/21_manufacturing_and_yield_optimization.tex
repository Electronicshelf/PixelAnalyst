\chapter{Manufacturing and Yield Optimization}

\section{Process Variations}

\begin{figure}[H]
\centering
\begin{tikzpicture}
  \begin{axis}[
    width=12cm,
    height=7cm,
    xlabel={Parameter Value},
    ylabel={Probability Density},
    grid=major,
    legend pos=north west,
    domain=-3:3,
    samples=100
  ]
  
  % Normal distribution
  \addplot[color=primarycolor, very thick] 
    {exp(-x^2/2)/sqrt(2*pi)};
  \addlegendentry{Target distribution}
  
  % Acceptance region
  \addplot[color=successcolor, fill=successcolor, fill opacity=0.2, domain=-1.5:1.5] 
    {exp(-x^2/2)/sqrt(2*pi)} \closedcycle;
  \addlegendentry{Acceptance region};
  
  % Rejection regions
  \addplot[color=accentcolor, fill=accentcolor, fill opacity=0.2, domain=-3:-1.5] 
    {exp(-x^2/2)/sqrt(2*pi)} \closedcycle;
  \addplot[color=accentcolor, fill=accentcolor, fill opacity=0.2, domain=1.5:3] 
    {exp(-x^2/2)/sqrt(2*pi)} \closedcycle;
  
  \end{axis}
\end{tikzpicture}
\caption{Process variation and yield. Tighter tolerances reduce yield and increase cost.}
\end{figure}

\section{Cost-Performance Trade-offs}

\begin{formulabox}{Yield Model}
\textbf{Area-limited yield:}
\begin{equation}
  Y = e^{-AD}
\end{equation}

where:
\begin{itemize}
  \item $Y$ = yield (fraction of good devices)
  \item $A$ = device area
  \item $D$ = defect density (defects per unit area)
\end{itemize}

\textbf{Implications:}
\begin{itemize}
  \item Larger displays have exponentially lower yields
  \item Higher resolution (more area) reduces yield
  \item Defect reduction is critical for profitability
\end{itemize}
\end{formulabox}

