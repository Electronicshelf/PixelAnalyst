\part{Color Science and Management}

\chapter{CIE Color Spaces}

\section{CIE XYZ Tristimulus System}
Given spectral power distribution $S(\lambda)$ and CIE 1931 color matching functions $\bar{x},\bar{y},\bar{z}$:
\begin{align}
X &= k\int S(\lambda)\bar{x}(\lambda)\,d\lambda \\
Y &= k\int S(\lambda)\bar{y}(\lambda)\,d\lambda \\
Z &= k\int S(\lambda)\bar{z}(\lambda)\,d\lambda
\end{align}
$Y$ is proportional to photometric luminance and is the bridge between colorimetry and brightness.

\begin{formulabox}{Chromaticity Coordinates}
\begin{equation}
x = \frac{X}{X+Y+Z},\quad y = \frac{Y}{X+Y+Z},\quad z = 1-x-y
\end{equation}
Chromaticity removes absolute luminance and keeps only hue/saturation information.
\end{formulabox}

\section{CIELAB Perceptual Space}

CIELAB improves perceptual uniformity relative to XYZ for color-difference work:
\begin{align}
L^* &= 116f\left(\frac{Y}{Y_n}\right)-16 \\
a^* &= 500\left[f\left(\frac{X}{X_n}\right)-f\left(\frac{Y}{Y_n}\right)\right] \\
b^* &= 200\left[f\left(\frac{Y}{Y_n}\right)-f\left(\frac{Z}{Z_n}\right)\right]
\end{align}
with
\begin{equation}
f(t)=
\begin{cases}
t^{1/3}, & t>\delta^3 \\
\frac{t}{3\delta^2} + \frac{4}{29}, & t\le\delta^3
\end{cases}
\quad\text{and}\quad \delta=\frac{6}{29}
\end{equation}

\begin{importantbox}{Engineering Note}
Most imaging pipelines still compute with linear RGB/XYZ, but quality gates (panel binning, print proofing, camera tuning) are often set in $L^*a^*b^*$ or $L^*u^*v^*$ because Euclidean distances in those spaces better approximate visibility.
\end{importantbox}

