\chapter{Chromatic Adaptation}

\section{Bradford Transform}
Chromatic adaptation maps tristimulus values from one illuminant to another while preserving perceived object color. The Bradford method is widely used in ICC workflows.

\begin{formulabox}{Bradford Adaptation Equation}
\begin{equation}
\mathbf{X}_D = M_B^{-1}
\begin{bmatrix}
\frac{L_D}{L_S} & 0 & 0 \\
0 & \frac{M_D}{M_S} & 0 \\
0 & 0 & \frac{S_D}{S_S}
\end{bmatrix}
M_B\mathbf{X}_S
\end{equation}
where $\mathbf{X}_S=[X\ Y\ Z]^T$ and:
\begin{equation}
M_B=
\begin{bmatrix}
0.8951 & 0.2664 & -0.1614 \\
-0.7502 & 1.7135 & 0.0367 \\
0.0389 & -0.0685 & 1.0296
\end{bmatrix}
\end{equation}
\end{formulabox}

\begin{examplebox}{Practical Use}
Common conversions include D65 $\leftrightarrow$ D50 when moving between display-referred workflows and ICC print profiles. Without adaptation, neutral grays drift warm/cool and color management fails even when primaries are correct.
\end{examplebox}

