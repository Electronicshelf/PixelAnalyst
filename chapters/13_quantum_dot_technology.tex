\chapter{Quantum Dot Technology}
\section{Working Principle}
Quantum dots (QDs) are semiconductor nanocrystals with size-tunable emission due to quantum confinement.

\begin{formulabox}{Brus Approximation (Confinement Shift)}
\begin{equation}
E_g(R) \approx E_{g,\text{bulk}} +
\frac{\hbar^2\pi^2}{2R^2}\left(\frac{1}{m_e^*}+\frac{1}{m_h^*}\right) -
\frac{1.8e^2}{4\pi\epsilon R}
\end{equation}
Then:
\begin{equation}
\lambda_{\text{peak}} \approx \frac{hc}{E_g(R)}
\end{equation}
\end{formulabox}

Smaller dots emit shorter wavelengths (blue shift), and larger dots emit longer wavelengths (red shift). Modern display films leverage narrow emission spectra (often 20--30 nm FWHM) for wider color gamut.

