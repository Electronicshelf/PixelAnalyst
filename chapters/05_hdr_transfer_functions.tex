\part{High Dynamic Range and Color}

\chapter{HDR Transfer Functions}
\section{Perceptual Quantizer (PQ / ST.2084)}

Perceptual Quantizer (PQ) is an \textbf{absolute} transfer function standardized in SMPTE ST.2084. It maps code values directly to absolute display luminance up to 10,000 nits and is designed so quantization steps track approximately constant visibility thresholds.

\begin{formulabox}{ST.2084 EOTF and Inverse}
\textbf{PQ EOTF (signal to luminance):}
\begin{equation}
L = 10000 \left[\frac{\max\left(V^{1/m_2}-c_1,0\right)}{c_2-c_3V^{1/m_2}}\right]^{1/m_1}
\end{equation}

\textbf{PQ inverse EOTF (luminance to signal):}
\begin{equation}
V = \left[\frac{c_1 + c_2\left(L/10000\right)^{m_1}}{1 + c_3\left(L/10000\right)^{m_1}}\right]^{m_2}
\end{equation}

with constants:
\begin{align}
m_1 &= \frac{2610}{16384} \approx 0.1593, &
m_2 &= \frac{2523}{32} \approx 78.8438 \\
c_1 &= \frac{3424}{4096} \approx 0.8359, &
c_2 &= \frac{2413}{128} \approx 18.8516, &
c_3 &= \frac{2392}{128} \approx 18.6875
\end{align}
\end{formulabox}

\begin{examplebox}{Typical 10-bit PQ Code Values}
Using ST.2084 inverse EOTF:
\begin{itemize}
  \item 100 nits: $V \approx 0.508$ (10-bit code $\approx 520$)
  \item 1000 nits: $V \approx 0.752$ (10-bit code $\approx 769$)
  \item 4000 nits: $V \approx 0.903$ (10-bit code $\approx 923$)
\end{itemize}
This non-linear allocation concentrates many code values in dark and mid tones where the eye is most sensitive.
\end{examplebox}

\section{Hybrid Log-Gamma (HLG / BT.2100)}

Hybrid Log-Gamma is a \textbf{relative} HDR transfer function designed for broadcast compatibility. It avoids static metadata and gracefully degrades on legacy SDR pipelines.

\begin{formulabox}{HLG OETF (scene linear to signal)}
For scene linear light $E$ normalized to reference white:
\begin{equation}
V = \begin{cases}
\sqrt{3E}, & 0 \le E \le \frac{1}{12} \\
a\ln(12E-b)+c, & \frac{1}{12} < E \le 1
\end{cases}
\end{equation}
where:
\begin{equation}
a = 0.17883277,\quad b = 0.28466892,\quad c = 0.55991073
\end{equation}
\end{formulabox}

\begin{definitionbox}{HLG System Gamma}
In practical displays, HLG output is shaped by a system gamma:
\begin{equation}
\gamma_{\text{sys}} \approx 1.2 + 0.42\log_{10}\left(\frac{L_W}{1000}\right)
\end{equation}
where $L_W$ is peak display luminance in nits. Brighter displays use higher effective contrast expansion.
\end{definitionbox}

\begin{table}[H]
\centering
\caption{PQ vs HLG in Production Workflows}
\label{tab:pq_hlg_comparison}
\begin{tabular}{@{}lll@{}}
\toprule
\textbf{Property} & \textbf{PQ (ST.2084)} & \textbf{HLG (BT.2100)} \\
\midrule
Signal type & Absolute luminance & Relative scene-referred \\
Metadata dependence & Common (MaxCLL/MaxFALL, dynamic) & None required \\
Best use case & Streaming, disc, mastering & Live broadcast \\
Backward SDR compatibility & Limited without tone mapping & Better graceful fallback \\
Peak coding target & Up to 10,000 nits & Display-dependent \\
\bottomrule
\end{tabular}
\end{table}


