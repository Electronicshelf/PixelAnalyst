\chapter{Advanced Colorimetry}

\section{Color Gamut Visualization}

\begin{figure}[H]
\centering
\begin{tikzpicture}[scale=6]
  % CIE 1931 chromaticity diagram (simplified)
  \fill[blue!20] (0.15,0.06) -- (0.64,0.33) -- (0.30,0.60) -- cycle;
  \draw[very thick, primarycolor] (0.15,0.06) -- (0.64,0.33) -- (0.30,0.60) -- cycle;
  \node at (0.36,0.33) {sRGB};
  
  \fill[green!20, opacity=0.5] (0.08,0.02) -- (0.68,0.32) -- (0.21,0.71) -- cycle;
  \draw[very thick, successcolor, dashed] (0.08,0.02) -- (0.68,0.32) -- (0.21,0.71) -- cycle;
  \node[green!50!black] at (0.32,0.45) {Rec.2020};
  
  \draw[->] (0,0) -- (0.8,0) node[right] {$x$};
  \draw[->] (0,0) -- (0,0.8) node[above] {$y$};
  
  \node[circle, fill=accentcolor, inner sep=1pt] at (0.3127,0.3290) {};
  \node[above right] at (0.3127,0.3290) {\small D65};
\end{tikzpicture}
\caption{CIE 1931 chromaticity diagram showing sRGB and Rec.2020 gamuts.}
\end{figure}

\section{RGB to XYZ Matrix Derivation}

\begin{formulabox}{Complete Derivation}
\textbf{Given:} RGB primary chromaticities and white point

\textbf{Step 1:} Construct primary matrix from chromaticities:
\begin{equation}
  P = \begin{bmatrix}
    x_r/y_r & x_g/y_g & x_b/y_b \\
    1 & 1 & 1 \\
    (1-x_r-y_r)/y_r & (1-x_g-y_g)/y_g & (1-x_b-y_b)/y_b
  \end{bmatrix}
\end{equation}

\textbf{Step 2:} White point in XYZ:
\begin{equation}
  \begin{bmatrix} X_w \\ Y_w \\ Z_w \end{bmatrix} = 
  \begin{bmatrix} x_w/y_w \\ 1 \\ (1-x_w-y_w)/y_w \end{bmatrix}
\end{equation}

\textbf{Step 3:} Scale factors:
\begin{equation}
  \begin{bmatrix} S_R \\ S_G \\ S_B \end{bmatrix} = 
  P^{-1} \begin{bmatrix} X_w \\ Y_w \\ Z_w \end{bmatrix}
\end{equation}

\textbf{Step 4:} Final RGB→XYZ matrix:
\begin{equation}
  M = P \cdot \text{diag}(S_R, S_G, S_B)
\end{equation}
\end{formulabox}

