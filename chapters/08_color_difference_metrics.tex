\chapter{Color Difference Metrics}

\section{Delta E Families}
\begin{formulabox}{Core Metrics}
\textbf{CIE76:}
\begin{equation}
\Delta E_{76} = \sqrt{(\Delta L^*)^2 + (\Delta a^*)^2 + (\Delta b^*)^2}
\end{equation}

\textbf{CIE94/CIEDE2000:} weighted forms that correct non-uniformity in chroma and hue regions:
\begin{equation}
\Delta E_{00} = \sqrt{
\left(\frac{\Delta L'}{k_LS_L}\right)^2 +
\left(\frac{\Delta C'}{k_CS_C}\right)^2 +
\left(\frac{\Delta H'}{k_HS_H}\right)^2 +
R_T\left(\frac{\Delta C'}{k_CS_C}\right)\left(\frac{\Delta H'}{k_HS_H}\right)}
\end{equation}
\end{formulabox}

\begin{table}[H]
\centering
\caption{Typical Interpretation of Color Differences}
\label{tab:deltae_thresholds}
\begin{tabular}{@{}ll@{}}
\toprule
\textbf{Delta E Range} & \textbf{Perceptual Meaning (Typical)} \\
\midrule
$< 1.0$ & Imperceptible in controlled viewing \\
$1.0$--$2.0$ & Barely perceptible to trained observers \\
$2.0$--$3.0$ & Perceptible, often acceptable in consumer displays \\
$> 3.0$ & Clearly visible mismatch \\
\bottomrule
\end{tabular}
\end{table}


