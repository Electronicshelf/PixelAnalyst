\chapter{Radiometry and Photometry}

\section{Radiometric Quantities}

Radiometry deals with the measurement of electromagnetic radiation from a purely physical perspective.

\begin{table}[H]
\centering
\caption{Fundamental Radiometric Quantities}
\label{tab:radiometric}
\begin{tabular}{@{}llll@{}}
\toprule
\textbf{Quantity} & \textbf{Symbol} & \textbf{Unit} & \textbf{Definition} \\
\midrule
Radiant Energy & $Q_e$ & joule (J) & Total EM energy \\
Radiant Flux & $\Phi_e$ & watt (W) & $dQ_e/dt$ \\
Radiant Intensity & $I_e$ & W/sr & $d\Phi_e/d\Omega$ \\
Radiance & $L_e$ & W/(sr·m²) & $d^2\Phi_e/(dA \cos\theta \, d\Omega)$ \\
Irradiance & $E_e$ & W/m² & $d\Phi_e/dA$ \\
\bottomrule
\end{tabular}
\end{table}

\subsection{Radiance: The Fundamental Display Quantity}

\begin{definitionbox}{Radiance}
Radiance $L_e$ is the flux per unit projected area per unit solid angle:
\begin{equation}
  L_e = \frac{d^2\Phi_e}{dA \cos\theta \, d\Omega}
\end{equation}

Key properties:
\begin{itemize}
  \item Describes brightness of a surface or light source
  \item Invariant along ray paths in lossless media (conservation of étendue)
  \item The quantity measured by cameras and seen by eyes
  \item THE fundamental quantity for characterizing display emission
\end{itemize}
\end{definitionbox}

\section{Photometric Quantities}

Photometry weights radiometric measurements by human visual sensitivity.

\begin{formulabox}{Photometric Conversion}
Photometric quantities are derived from radiometric ones via:
\begin{equation}
  X_v = K_m \int_{\lambda} X_{e,\lambda}(\lambda) \, V(\lambda) \, d\lambda
\end{equation}
where:
\begin{itemize}
  \item $K_m = 683$ lm/W at 555 nm (peak photopic sensitivity)
  \item $V(\lambda)$ = photopic luminous efficiency function
  \item Integration typically over 380--780 nm
\end{itemize}
\end{formulabox}

\begin{figure}[H]
\centering
\begin{tikzpicture}
  \begin{axis}[
    width=12cm,
    height=7cm,
    xlabel={Wavelength (nm)},
    ylabel={Relative Luminous Efficiency $V(\lambda)$},
    xmin=380, xmax=780,
    ymin=0, ymax=1.1,
    grid=major,
    legend pos=north west
  ]
  
  % Photopic (day vision)
  \addplot[color=warningcolor, very thick, smooth, domain=380:780] 
    {exp(-((x-555)/80)^2)};
  \addlegendentry{Photopic $V(\lambda)$ (day)}
  
  % Scotopic (night vision) 
  \addplot[color=primarycolor, dashed, thick, smooth, domain=380:650] 
    {exp(-((x-507)/75)^2)};
  \addlegendentry{Scotopic $V'(\lambda)$ (night)}
  
  \end{axis}
\end{tikzpicture}
\caption{CIE photopic and scotopic luminous efficiency functions. Note the blue shift in scotopic vision (Purkinje effect).}
\label{fig:luminous_efficiency}
\end{figure}

\subsection{Luminance: Display Brightness}

\begin{importantbox}{Critical Distinction}
\textbf{Luminance} (cd/m² = nits) describes light \textit{emitted} from a display surface.

\textbf{Illuminance} (lux = lm/m²) describes light \textit{incident} on a surface.

Confusing these is a common error! A display with 500 nits luminance in a room with 500 lux illuminance does NOT mean they are equal -- they have different units and meanings.
\end{importantbox}

\begin{table}[H]
\centering
\caption{Typical Luminance and Illuminance Values}
\label{tab:typical_values}
\begin{tabular}{@{}lcc@{}}
\toprule
\textbf{Condition/Device} & \textbf{Luminance (nits)} & \textbf{Illuminance (lux)} \\
\midrule
\multicolumn{3}{l}{\textit{Display Luminance (emitted):}} \\
Cinema DCI projector & 48 & --- \\
Laptop display (typical) & 300--500 & --- \\
Smartphone (typical) & 500--1000 & --- \\
HDR TV (peak) & 1000--4000 & --- \\
\midrule
\multicolumn{3}{l}{\textit{Ambient Illuminance (incident):}} \\
Moonlight & --- & 0.1 \\
Home lighting & --- & 100 \\
Office lighting & --- & 500 \\
Overcast day & --- & 10,000 \\
Direct sunlight & --- & 100,000 \\
\bottomrule
\end{tabular}
\end{table}

\subsection{Lambertian Reflection Model}

\begin{formulabox}{Reflection from Display Surface}
For a Lambertian (ideal diffuse) reflector with reflectance $\rho$ under illuminance $E_{\text{ambient}}$:
\begin{equation}
  L_{\text{reflected}} = \frac{E_{\text{ambient}} \times \rho}{\pi} \quad [\text{cd/m}^2]
\end{equation}

The $\pi$ factor arises from integrating Lambertian emission over the hemisphere.

\textbf{Typical display reflectances:}
\begin{itemize}
  \item Glossy display: $\rho \approx 0.02$ to $0.04$ (2--4\%)
  \item Anti-glare coating: $\rho \approx 0.01$ to $0.02$ (1--2\%)
  \item Matte screen protector: $\rho \approx 0.05$ to $0.10$ (5--10\%)
\end{itemize}
\end{formulabox}

\begin{examplebox}{Practical Example: Office Environment}
Office with $E = 500$ lux, glossy display with $\rho = 0.04$:
\begin{equation}
  L_{\text{reflected}} = \frac{500 \times 0.04}{\pi} \approx 6.4 \text{ nits}
\end{equation}

This 6.4 nits adds to the display's black level, degrading contrast!

For an OLED with native black level of 0.0005 nits:
\begin{itemize}
  \item Dark room contrast: $1000 / 0.0005 = 2{,}000{,}000:1$
  \item Office contrast: $1000 / 6.4 = 156:1$
\end{itemize}

The ``infinite'' OLED contrast becomes 156:1 in typical office lighting.
\end{examplebox}

% ============================================================================
% PART II: PIXEL ARCHITECTURE
% ============================================================================

