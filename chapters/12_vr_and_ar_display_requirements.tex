\chapter{VR and AR Display Requirements}
\section{Image Sharpness and Motion Requirements}
\begin{formulabox}{Pixels-Per-Degree and Latency Budget}
\textbf{Angular sampling density:}
\begin{equation}
\text{PPD} = \frac{N_{\text{horizontal}}}{\text{FOV}_{\text{horizontal}}}
\end{equation}

\textbf{Motion-to-photon latency:}
\begin{equation}
t_{\text{MTP}} = t_{\text{tracking}} + t_{\text{render}} + t_{\text{scanout}} + t_{\text{pixel}}
\end{equation}
\end{formulabox}

\begin{examplebox}{Why PPD Matters}
A headset with 2160 horizontal pixels and 100 degree horizontal FOV yields:
\begin{equation}
\text{PPD} = 2160/100 = 21.6
\end{equation}
This is significantly below foveal vision limits, which explains visible aliasing and the push toward micro-OLED and micro-LED microdisplays.
\end{examplebox}

\section{Minimum Specifications}
\begin{itemize}
\item Resolution: 1500+ PPI panel class for compact optics
\item Refresh rate: 90+ Hz (120+ preferred)
\item Persistence: <3 ms (reduce display-induced blur)
\item Latency: <20 ms (motion-to-photon)
\item Field of view: 90--110 degrees
\item IPD adjustment: 58--72 mm range
\end{itemize}

